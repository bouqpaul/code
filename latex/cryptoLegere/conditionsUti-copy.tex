\documentclass{article}
\usepackage[utf8]{inputenc}
\usepackage[T1]{fontenc}

\begin{document}

Shift from general-purpose computers to dedicated resource-constrained devices
The shift from desktop computers to small devices brings a wide range of new security and privacy concerns.

Lack of crypto standards that are suitable for constrained devices
There are several emerging areas in which highly constrained devices are interconnected, working in concert to accomplish some task.
Examples of these areas include: automotive systems, sensor networks,  healthcare,  distributed  control  systems,
the  Internet  of  Things (IoT),  cyber-physical systems,  and  the  smart  grid.
On the lower end of the spectrum are devices such as embedded systems, RFID devices and sensor networks.
Lightweight cryptography is primarily focused on the highly-constrained devices that can be found at this end of the spectrum.

pas assez de puissance de calcul/énergie dispo => temps de calcul/energie dépensée trop important.
on cherche des algo avec peu de GE mais aussi timing and power requirements.

Of course it’s hard to get a handle on block cipher performance on devices that don’t yet exist



\section{Chiffrement de flux}

Stream Ciphers
A5/1 = 2G GSM protocol still uses this algorithm
Satellite phones
Crypto-1 It is a stream cipher used by the Mifare classic line of smartcards of nxp

Content Scrambling System (Css) In order to implement Digital Rights Managements (DRMs), the content of dvd discs is encrypted.
This encryption used to be performed with a stream cipher called Css
Csa-SC digital television broadcas

\section{Chiffremment par bloc}

Depuis 2012, pour permettre son utilisation au sein du gouvernement américain et leurs différentes agences et permet d'avoir des niveaux acceptable de sécurité.
optimize(for software applications) by NSA for IOT (Internet Of Things)
add-rotate-xor cipher
chiffrage par bloc
Pas de standarsisation par l'ISO, pas de confiance en la NSA, preuves partielles de faiblesse
La NSA a déjà poussé la standardisation d'un algo de chiffrage avec des "backdoor" (Dual EC DRBG).

Simon and Speck are lightweight block cipher families developed by the U.S. National Security Agency for high performance in constrained hardware and software environment

Simon and Speck have compelling advantages for high-throughput ASIC applications
Of course it’s hard to get a handle on block cipher performance on devices that don’t yet exist
Peut être implémenté sur des microcontrolleurs 8 bits.
Le but de l'algo Speck est d'obtenir des performances et un niveau de sécurité acceptable.
Il a été optimisé pour une application logicielle et matérielle.
constrained devices
NIST is considering applications of lightweight cryptography for sensor networks, healthcare and the smart grid.
NASA has expanding programs for small satellites such as CubeSats which may need lightweight algorithms.
Finally, Linux added SPECK support for efficient, opportunistic encryption on low-resource devices to the kernel in February 2018.

Simon and Speck are also unique among existing lightweight block ciphers in their support for a broad 12 range of block and key sizes, allowing the cryptography to be precisely tuned to a particular application.

This can involve optimizing with respect to the instruction set for a certain microcontroller, or designing algorithms for a particular ASIC application (e.g., with hard-wired key or for IC printing), or designing specifically for low-latency applications, and so on.


Simon and Speck aim to be the sort of generalist block ciphers that we think will be required for future applications in the IoT era.

Many of the sensors, actuators and other micromachines that will function as eyes, ears and hands in IoT networks.


Block Ciphers
Cryptomeria “C2” in the literature to be used by things”,
namely dvd players (in which case it can be seen as a successor of Css) and some SD cards

\section{Conclusion}
Moreover, as noted in [11], many of these devices will interact with a backend server, so a lightweight block cipher should also perform well on 64-bit processors


\end{document}
