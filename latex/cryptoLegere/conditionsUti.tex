\documentclass{article}
\usepackage[utf8]{inputenc}
\usepackage[T1]{fontenc}

\begin{document}
Depuis 2012, pour permettre son utilisation au sein du gouvernement américain et leurs différentes agences et permet d'avoir des niveaux acceptable de sécurité.
optimize(for software applications) by NSA for IOT (Internet Of Things)
add-rotate-xor cipher
chiffrage par bloc
Pas de standarsisation par l'ISO, pas de confiance en la NSA, preuves partielles de faiblesse
La NSA a déjà poussé la standardisation d'un algo de chiffrage avec des "backdoor" (Dual EC DRBG).

Simon and Speck are lightweight block cipher families developed by the U.S. National Security Agency for high performance in constrained hardware and software environment

\section{Domaines d'appllications}

\subsection{Speck}
Simon and Speck have compelling advantages for high-throughput ASIC applications
Of course it’s hard to get a handle on block cipher performance on devices that don’t yet exist
Peut être implémenté sur des microcontrolleurs 8 bits.
Le but de l'algo Speck est d'obtenir des performances et un niveau de sécurité acceptable.
Il a été optimisé pour une application logicielle et matérielle.
constrained devices
NIST is considering applications of lightweight cryptography for sensor networks, healthcare and the smart grid.
NASA has expanding programs for small satellites such as CubeSats which may need lightweight algorithms. Finally, Linux added SPECK support for efficient, opportunistic encryption on low-resource devices to the kernel in February 2018.

Moreover, as noted in [11], many of these devices will interact with a backend server, so a lightweight block cipher should also perform well on 64-bit processors

Simon and Speck are also unique among existing lightweight block ciphers in their support for a broad 12 range of block and key sizes, allowing the cryptography to be precisely tuned to a particular application.

This can involve optimizing with respect to the instruction set for a certain microcontroller, or designing algorithms for a particular ASIC application (e.g., with hard-wired key or for IC printing), or designing specifically for low-latency applications, and so on.


Simon and Speck aim to be the sort of generalist block ciphers that we think will be required for future applications in the IoT era.

In 2011, prompted by potential U.S. government requirements for lightweight ciphers (e.g., SCADA and logistics applications) and the concerns with existing cryptographic solutions which we’ve noted above, we began work on the Simon and Speck block cipher families on behalf of the Research Directorate of the U.S. National Security Agency (NSA)


Many of the sensors, actuators and other micromachines that will function as eyes, ears and hands in IoT networks.

\subsection{Spongent}






\end{document}
